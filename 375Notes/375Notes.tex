\documentclass[14pt]{article}

\usepackage[margin=1in]{geometry}
\usepackage{amsmath}
\usepackage{amssymb}
\usepackage{hyperref}
\usepackage{multicol}
\usepackage{float}
\usepackage{fancyhdr}
\usepackage{graphicx}
\usepackage{xcolor}
\usepackage{chngcntr}

\renewcommand{\familydefault}{\sfdefault}
\counterwithin*{equation}{section}
\counterwithin*{equation}{subsection}
\parindent 0ex
\everymath{\displaystyle{}}

\title{Math 375\\Differential Equations for Engineers and Scientists}
\author{Andy Smit}
\date{Spring 2019}

\begin{document}
    \maketitle
    \section{Differential Equations}
    Any equation that relates a quantity to its rate of changes.
    A solution for a differential equation is any formula that makes the
    equation true.
    \subsection{Classification of Differential Equations}
    There is two methods by which to classify differential equations.
    The first method to classify differential equations is to look at
    the highest order derivative in the equation.
    The largest $n$ such that the equation has an $n^{th}$ order
    derivative is an $n^{th}$ order differential equation.
    The second method to classify differential equations is by the
    number of variables that are differentiated with respect to.
    If a differential equation involves derivatives with respect to a
    single variable, it is a ordinary differential equation.
    If a differential equation involves derivatives with respect to
    multiple variable, it is a partial differential equation.
    \section{First Order Differential Equations}
    \section{Second Order Linear Differential Equations}
    \subsection{Separable Equations:}
    A separable equation is a differential equation that can be written 
    as
    \begin{equation}\label{seperable_equation}
        \frac{dy}{dx}=f(x)\,g(y)
    \end{equation}
    where the $x$ and $y$ are separated into factors.
    \eqref{seperable_equation} can be rearranged to obtain
    $$\frac{1}{g(y)}\frac{dy}{dx}=f(x)$$
    Now both sides can be integrated with respect to $x$ to solve
    $$\int\frac{1}{g(y)}\frac{dy}{dx}\,dx=\int f(x)\,dx+C$$
    \subsubsection{An Important Separable Equation}
    Consider the differential equation with the initial value,
    \begin{equation}\label{exponential_growth}
        \frac{dy}{dx}=ky,\ y(0)=y_0
    \end{equation}
    \section{Systems of Differential Equations}
    \section{Fourier Series and Boundary Value Problems}
    \section{Laplace Transform}
    \section{Applications of Differential Equations}

\end{document}