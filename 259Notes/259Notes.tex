\documentclass[14pt]{article}
\usepackage[margin=1in]{geometry}
\usepackage{amsmath}
\usepackage{amssymb}
\usepackage{graphicx}
\usepackage{xcolor}
\renewcommand{\familydefault}{\sfdefault}
\parindent 0ex
\everymath{\displaystyle{}}

\begin{document}
    \begin{center}
        \textbf{PHYS 259: Notes\\Electricity And Magnetism\\Andy Smit}
    \end{center}
    \textbf{Coulomb's Law:}\\
    The magnitude of the force between 2 charges is proportional to the product of the charges on the 2 particles over the distance, $r$, squared or,  $$F_{1\ on\ 2}=F_{2\ on\ 1}=K \frac{|q_1||q_2|}{r^2}$$where $K$ is the electrostatic constant and $K=8.99\times 10^9 N\cdot m^2 C^{-2}$ or equaly $$F_{1\ on\ 2}=F_{2\ on\ 1}=\frac{1}{4\pi\epsilon_0}\frac{|q_1||q_2|}{r^2}$$ where $\epsilon_0$ is the permittivity of free space and $\epsilon_0=8.85\times10^-12 s^4\cdot A^2\cdot m^{-3}\cdot kg^{-1}$\\\\
    \textbf{Superposition Principle:}\\
    The total force on a charge is the vector sum of the individual forces acting on it. $$\vec{F}_{on\ n}=\sum\limits_{k=1}^{n-1}\vec{F}_{k\ on\ n}$$
    $E=\int dE\cos\theta=\int\limits_0^{2\pi r}dt$\\
    \textbf{Gauss' Law:}\\
    The 
    $\oint\vec{E} \cdot\vec{n}dA$\\
    \textbf{Capacitors: }\\
    A capacitor is any two electrodes separated by some distance. Regardless of the geometry we call the electrodes "plates".
    By convention a capacitor has equal and opposite charges on its plates, although technically this does not have to be true.
    The source charges on the capacitor plates create a uniform electric field between the plates of 
    $$\vec E=\frac{\sigma}{\epsilon_0}$$
    The capacitance $C$ of two parallel plates is given by $\frac{\epsilon_0A}{d}$ where
    $$Q=C\Delta V$$
\end{document}